
\subsubsection{Reconstruction of $Q^2$ and $x$}\label{sec:ana_q2}

The four-momentum transfer squared is 
\begin{equation} \label{eq:qsq}
Q^2 \hskip 0.05in
= \hskip 0.05in 2 \hskip 0.02in E \hskip 0.02in 
E^{\prime} \hskip 0.02in (1- {\rm cos}(\theta))
\end{equation}
where $E$ is the incident energy, $E^{\prime}$ is the
final momentum or energy of the 
electron ($E^{\prime} \gg m_e$) and
$\theta$ is the scattering angle.  

For the beam energy we used the Tiefenbach energy (need to 
explain this) of ??? GeV
and assumed a 3 MeV (???) average energy loss to the center of the 
target which is applied
this as a correction to the beam energy.  
The error in the beam energy $E$ and $E^{\prime}$ are assumed
conservatively to be 3 MeV based on a history of these measurements
in Hall A.  The most important error is in $\theta$ ...

Perhaps need a table of errors.

%the following is copied from Bob's v1 but according to the new outline should be separated into two simulations. The HAMC should go into section ``Q2 reconstruction'' (above), the HATS should be a stand-alone subsection of ``Analysis''.

\subsection{Simulation}

Two simulation packages were used to support the analysis of this experiment.
The package called ``hamc'' (Hall A Monte Carlo) was used to simulate
the events and the spectrometer acceptance, while a second package
called ``hats'' (Hall A Trigger Simulation) was used to simulate the
response of the trigger used to identify electrons and pions, providing
a calculation of our deadtime.

In ``hamc'', events are generated using a physics class that has information about the cross section and asymmetry.  
The tracks are generated uniformly in solid angle 
\hskip 0.05in $d\Omega = sin(\theta) \hskip 0.02in d\theta \hskip 0.02in d\phi$ \hskip 0.05in and 
the results later weighted by the differential cross section $\frac{d\sigma}{d\Omega}$. 
The simulated tracks undergo multiple scattering in the target and 
energy loss from the target from external and internal Brehmstrahlung as 
well as ionization loss, 

The generated four-vectors are transported to the detector in the HRS focal plane using 
a set of polynomials that model the trajectories of electrons through the magnetic fields.
The beam raster is simulated, which produces a smearing of the beam on target.
The events are transported to intermediate apertures such as the collimator 
or the entrance to quadrupoles. 
Events that reach the HRS focal plane and intersect the detectors are integrated 
to compute the total rate and average asymmetry.

Here describe ``hats'' ...

