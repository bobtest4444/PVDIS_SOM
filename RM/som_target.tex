\subsection{Target}\label{sec:target}

The Hall A cryogenic target system 
\cite{A-NIM} was used for this experiment. 
We used the a 20 cm deuterium target cell.
The cell sits in an evacuated scattering chamber, 
along with subsystems for cooling,
temperature and pressure monitoring, target motion, gas-handling,
controls, and a solid and dummy target ladder.  

The liquid deuterium loop was operated at a temperature of 19 K and a
pressure of $\sim 26$ psia, leading to a density of about 0.0723
g/cm$^3$. The Al-walled target cells were 6.48 cm in diameter, and
were oriented horizontally, along the beam direction. The upstream
window thickness was 0.071 mm, the downstream window thickness was
0.094 mm, and the side wall thickness was 0.18 mm. Also mounted on the
target ladder were solid thin targets of carbon, and aluminum dummy
target cells, for use in background and spectrometer studies.

The target was mounted in a cylindrical scattering chamber of 104 cm
diameter, centered on the pivot for the spectrometers. The scattering
chamber was maintained under a $10^{-6}$ torr vacuum. The
spectrometers view exit windows in the scattering chamber that were
made of 0.406 mm thick Al foil.

To spread the heat load on the the target end-cap,
the beam was rastered at 20 kHz by two sets of
steering magnets 23 m upstream of the target. These magnets deflected
the beam by up to $\pm 2.5$ mm in $x$ and $y$ at
the target.  
Local target boiling would manifest itself as an increase in fluctuations in the
measured scattering rate, which would lead to an increase in the
standard deviation of the pulse-pair asymmetries in the data, above
that expected from counting statistics. Studies of the pulse-pair
asymmetries for various beam currents and raster sizes were performed,
at a lower $Q^2$ and thus at a higher scattering rate. Figure
\ref{fig6_gar_boil_kk} shows the standard deviation of the 
pulse-pair asymmetries, extrapolated to full current values, for
various beam currents and raster sizes. A significant increase over
pure counting statistics, indicating local boiling effects, was
observed only for the combination of a small raster (1.0 mm) size and
large beam current (94 $\mu$A).  During the experiment we used
larger raster sizes for which there was little boiling noise.
