
\subsection{Polarized Electron Beam}\label{sec:app_electronbeam}

The electron beam originated from a GaAs photocathode illuminated by 
circularly polarized light~\cite{Sinclair2007}.
By reversing the sign of the laser circular 
polarization, the direction of the spin at the target could be 
reversed rapidly \cite{Paschke:2007zz}.
A half-wave ($\lambda$/2) plate was periodically inserted into the 
laser optical path which passively reversed the
sign of the electron beam polarization. 
Roughly equal statistics were thus 
accumulated with opposite signs for the measured asymmetry, which suppressed 
many systematic effects.  
The direction of the polarization could be
controlled by a Wien filter and solenoidal lenses
near the injector \cite{GramesWien2011}.  The accelerated beam was 
directed into Hall A, where its intensity, energy and trajectory on 
target were inferred from the response of several monitoring devices.

Each period of constant spin direction is referred to as a ``window''.
The beam monitors, target, detector components and 
electronics were designed so that
the fluctuations in the fractional difference in the PMT response between
a pair of successive windows were
dominated by scattered electron counting statistics.
To keep spurious beam-induced asymmetries under
 control at well below the ppm level, 
careful attention was given to the design and configuration of the laser 
optics leading to the photocathode \cite{Paschke:2007zz}.

The spin-reversal rate was 30 Hz.
The integrated response of each detector PMT and beam monitor
was digitized and recorded for each window.
The raw spin-direction asymmetry $A_{raw}$
in each spectrometer arm was computed from the the detector response
normalized to the beam intensity 
for each window pair. 
The sequence of these
patterns was chosen with a pseudorandom number generator.

