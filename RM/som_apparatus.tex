
\section{Apparatus}\label{sec:apparatus}

\par The experimental techniques for measuring small 
asymmetries of order 1 ppm have been successfully deployed in
parity experiments at electron
scattering facilities \cite{SLAC}-\cite{happex}.
The recent experiments at Jefferson Lab, such as HAPPEX ~\cite{happex}
and PREX ~\cite{prex} have maintain systematic errors associated with helicity
reversal at the $10^{-8}$ level.
The asymmetries sought for in this experiment are of order 100 ppm with
accuracies of about 1 ppm, which is two orders-of-magnitude above the 
established systematic error.

A significant challenge of the measurement 
is to separate electrons from the charged pion background that arise from electro- or photo-productions. 
While the standard HRS detector package and data acquisition (DAQ) system routinely provide 
such a high particle identification (PID) performance, they are based on full recording 
of the detector signals and are limited to event rates up to 4 kHz.
This is not sufficient for the few-hundred kHz rates for the experiment. 
Thus we have built new DAQ designed to count event rates up to 1~MHz with hardware-based 
particle identification ~\cite{pvdis_nim}.

The main parts of the apparatus will be described in this section.
These include the polarized electron beam, the beam monitors, the spectrometers
and detectors, the data acquisition system, and the beam polarimeters.
